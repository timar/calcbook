\section*{Előszó a 2. kiadáshoz}
\thispagestyle{empty}
\small
Ebből a könyvből az OpenOffice.org táblázatkezelőjének, a Calcnak a használatát lehet elsajátítani. Az anyag teljes mértékben lefedi mind az érettségi, mind az ECDL táblázatkezelő moduljának a témaköreit. A tanulást több, mint 150 szemléletes kép könnyíti meg, illetve 35 gyakorló feladat segít az ismeretek elmélyítésében. A 2. kiadásra azért került sor, mert a Calc munkalapfüggvényeinek nevei az OpenOffice.org újabb verzióiban már magyarul vannak, ugyanúgy, ahogy az oktatásban és a munkahelyeken elterjedt magyar nyelvű Microsoft Excelben, ezért a függvényneveket magyarra kellett fordítani a könyv szövegében és ábráiban is. Egyúttal jónéhány sajtóhibát is sikerült javítani.

\smallskip

Az OpenOffice.org egy teljes körű irodai alkalmazáscsomag szövegszerkesztéshez, táblázatkezeléshez, bemutatók és illusztrációk készítéséhez, adatbázisok használatához és egyéb feladatokhoz. Előnyei között említhetjük, hogy több nyelven (kb. 70) és több platformon (Windows, Linux, Mac OS X stb.) elérhető, nemzetközileg szabványosított formátumban tárolja az adatokat, valamint írja és olvassa a Microsoft Office állományait. Letöltése és használata bármilyen célra – beleértve az üzleti alkalmazást is – teljesen ingyenes. Ennek köszönhetően az egész világon és Magyarországon is számos állami szervezet, vállalkozás és magánszemély tért át vagy tér át a használatára, illetve tervezi az áttérést a közeljövőben.

\smallskip

Az OpenOffice.org története 1986-ban kezdődött, ekkor kezdte el fejleszteni egy német cég, a Star Division a StarWriter nevű szövegszerkesztőt az akkoriban elterjedt DOS platformra. 1993-ban megszületett a termék windowsos verziója, melyet egy évvel később az OS/2-es és a macintoshos verzió követett. 1995-ben a StarOffice nevet vette fel a termék, ekkor már több jelentős komponenst tartalmazott: szövegszerkesztőt (StarWriter), egyszerű rajzprogramot (StarImage), táblázatkezelőt (StarCalc), grafikonkészítőt (StarChart) és egy vektoros rajzolóprogramot (StarDraw). A későbbi változatok már böngészőt és HTML-szerkesztőt, bemutatókészítőt (StarImpress) és adatbázis-kezelőt (StarBase) is tartalmaztak.

\smallskip

A StarDivision története 1999-ben ért véget, amikor a Sun felvásárolta a céget. Simon Phipps, volt Sun-alkalmazott szerint „{\em a StarDivision felvásárlásának legfontosabb oka az volt, hogy abban az időben a Sun alkalmazottainak száma elérte a 42 ezret és minden munkatárs rendelkezett egy Unix-munkaállomással és egy windowsos laptoppal. Olcsóbb volt megvenni egy céget, amely irodai alkalmazást fejlesztett Solaris és Linux operációs rendszerre, mint 42 ezer Microsoft Office licencet venni a Microsofttól.}” A StarOffice 5.2-es verzióját a Sun ingyenesen letölthetővé tette, hogy így próbálja meg növelni a termék piaci részesedését. A későbbi változatok már fizetős, kereskedelmi termékekként kerültek a felhasználókhoz.

\smallskip

A szabad szoftveres közösség számára a „nagy nap” 2000. október 13-án jött el, amikor a Sun OpenOffice.org néven szabaddá tette az irodai csomag forráskódját. Több, harmadik fél által készített, licencelt komponenst ki kellett venni, illetve szükség volt több átalakításra is, mielőtt megszülethetett volna az OpenOffice.org kiindulási forrása. Az OpenOffice.org körülbelül másfél év alatt érte el az első nagy mérföldkövet: az 1.0-s verzió 2002. május elsején jelent meg.

\smallskip

A közelmúltig a Sun volt az OpenOffice.org legnagyobb támogatója és a fejlesztés vezetője. 2010-ben zárult le a Sun felvásárlása az Oracle által, de ez nem okoz változást. Az Oracle átvette a fejlesztőket, továbbra is fejleszti és támogatja a nyílt forrású OpenOffice.org-ot, mint a közösség legjelentősebb tagja. A StarOffice Oracle Open Office néven él tovább.

\smallskip


2002. február 1-től 4-ig, egy maratoni „fordítóbuli” keretén belül készült el az OpenOffice.org irodai programcsomag magyarul beszélő változata. A hivatalos bemutatóra 2002. február 23-án került sor. A munkában mintegy 150 ember vett részt. Ez a munka teremtette meg a lehetőségét minden további fejlesztésnek, és ez az esemény vezetett el az \href{http://www.fsf.hu}{FSF.hu Alapítvány} megalapításához is. A magyar OpenOffice.org elkészítését azóta is az FSF.hu Alapítvány koordinálja. 2003 során tovább folyt a közösségi fordítói munka, februárban a súgóból készültek el részek, novemberben pedig a részletes tippek lettek lefordítva. A súgó fordítása 2005-re lett kész. Azóta csak az új verziókban megjelenő módosítások és újdonságok lefordítása, valamint a fordítás folyamatos javítgatása ad feladatot.

\smallskip

A magyar OpenOffice.org-gal kapcsolatos aktuális hírek és információk a \url{http://hu.openoffice.org/} honlapon olvashatók.
\vfill
\begin{flushright}
Tímár András\\
szoftverhonosító\\
OpenOffice.org
\end{flushright}
\normalsize


